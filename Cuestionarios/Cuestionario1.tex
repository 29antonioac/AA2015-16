\documentclass[a4paper, 11pt]{article}

%Comandos para configurar el idioma
\usepackage[spanish,activeacute]{babel}
\usepackage[utf8]{inputenc}
\usepackage[T1]{fontenc} %Necesario para el uso de las comillas latinas.
\usepackage{geometry} % Used to adjust the document margins

%Importante que esta sea la última órden del preámbulo
\usepackage{hyperref}
   \hypersetup{
     pdftitle={Cuestionario de teoría 1},
     pdfauthor={Antonio Álvarez Caballero},
     unicode,
     breaklinks=true,  % so long urls are correctly broken across lines
     colorlinks=true,
     urlcolor=blue,
     linkcolor=darkorange,
     citecolor=darkgreen,
     }

   % Slightly bigger margins than the latex defaults

   \geometry{verbose,tmargin=1in,bmargin=1in,lmargin=1in,rmargin=1in}
\newcommand\fnurl[2]{%
  \href{#2}{#1}\footnote{\url{#2}}%
}


%Paquetes matemáticos
\usepackage{amsmath,amsfonts,amsthm}
\usepackage[all]{xy} %Para diagramas
\usepackage{enumerate} %Personalización de enumeraciones
\usepackage{tikz} %Dibujos
\usepackage{ dsfont }

%Tipografía escalable
\usepackage{lmodern}
%Legibilidad
\usepackage{microtype}

%Código
\usepackage{listings}
\usepackage{color}

\definecolor{dkgreen}{rgb}{0,0.6,0}
\definecolor{gray}{rgb}{0.5,0.5,0.5}
\definecolor{mauve}{rgb}{0.58,0,0.82}

\lstset{frame=tb,
  language=Python,
  aboveskip=3mm,
  belowskip=3mm,
  showstringspaces=false,
  columns=flexible,
  basicstyle={\small\ttfamily},
  numbers=left,
  numberstyle=\tiny\color{gray},
  keywordstyle=\color{blue},
  commentstyle=\color{dkgreen},
  stringstyle=\color{mauve},
  breaklines=true,
  breakatwhitespace=true,
  tabsize=3
}

\title{Cuestionario de teoría 1}
\author{Antonio Álvarez Caballero\\
    \href{mailto:analca3@correo.ugr.es}{analca3@correo.ugr.es}}
\date{\today}

\theoremstyle{definition}
\newtheorem{ejercicio}{Ejercicio}
\newtheorem{cuestion}{Cuestión}
\newtheorem*{solucion}{Solución}
\newtheorem*{bonus}{BONUS}



%%%%%%%% New sqrt
\usepackage{letltxmacro}
\makeatletter
\let\oldr@@t\r@@t
\def\r@@t#1#2{%
\setbox0=\hbox{$\oldr@@t#1{#2\,}$}\dimen0=\ht0
\advance\dimen0-0.2\ht0
\setbox2=\hbox{\vrule height\ht0 depth -\dimen0}%
{\box0\lower0.4pt\box2}}
\LetLtxMacro{\oldsqrt}{\sqrt}
\renewcommand*{\sqrt}[2][\ ]{\oldsqrt[#1]{#2} }
\makeatother

%%%%%%%%%%%%%%%%%%%%%%%%%%%%%%%%%%%%%%%%%%%%%

%%%%%%%%%%%% Norm
\newcommand{\norm}[1]{\left\lVert#1\right\rVert}
%%%%%%%%%%%%%%%%%

%%%%%%%%%%% Ceil
\usepackage{mathtools}
\DeclarePairedDelimiter{\ceil}{\lceil}{\rceil}
%%%%%%%%%%%%%%%%%%%%%%%%%%%%%%%%%%%%%%%%%%%

\begin{document}

  \maketitle

  \section{Cuestiones}

  \begin{cuestion}
    Identificar, para cada una de las siguiente tareas, qué tipo de aprendizaje automático es el adecuado (supervisado, no supervisado, por refuerzo) y los datos de aprendizaje que deberíamos usar. Si una tarea se ajusta a más de un tipo, explicar cómo y describir los datos para cada tipo.
    \begin{enumerate}
      \item[a)] Categorizar un grupo de animales vertebrados en pájaros, mamíferos, reptiles, aves y anfibios.
      \item[b)] Clasificación automática de cartas por distrito postal.
      \item[c)] Decidir si un determinado índice del mercado de valores subirá o bajará dentro de un periodo de tiempo fijado.
    \end{enumerate}

  \end{cuestion}

  \begin{solucion}
    Los tipos de aprendizaje son:
    \begin{enumerate}
      \item[a)] Aprendizaje supervisado. Lo importante aquí es el conjunto de datos, ya que cada uno de estas clases tiene características bastante bien definidas (pelo, plumas, número de patas, número de alas, piel seca o húmeda, escamas, órganos respiratorios, \ldots). Por esta misma razón estaría bien intentar realizar una aproximación por diseño.
      \item[b)] Aprendizaje no supervisado. Aquí no hace falta que se sepa realmente la etiqueta de los datos, simplemente con que el sistema sea capaz de juntar las cartas que pertenezcan al mismo distrito postal
      \item[c)] Por refuerzo. No siempre podemos basarnos en datos que ya tenemos para decidir esto (los datos de hace 4 años no tienen por qué ser relevantes ahora), por lo que una aproximación por refuerzo irá aprendiendo sobre la marcha con la actividad actual del mercado de valores.
    \end{enumerate}
  \end{solucion}

  \begin{cuestion}
  ¿Cuáles de los siguientes problemas son más adecuados para una aproximación por aprendizaje y cuáles más adecuados para una aproximación por diseño? Justificar la decisión.
  \begin{enumerate}
  \item[a)] Determinar el ciclo óptimo para las luces de los semáforos en un cruce con mucho tráfico.
  \item[b)] Determinar los ingresos medios de una persona a partir de sus datos de nivel de educación, edad, experiencia y estatus social.
  \item[c)] Determinar si se debe aplicar una campaña de vacunación contra una enfermedad.
  \end{enumerate}

  \end{cuestion}

  \begin{solucion}
    Las aproximaciones son:
    \begin{enumerate}
      \item[a)] Aproximación por diseño. Este es un problema que puede llegar a ser crítico si se produce algún fallo, luego una aproximación por diseño (exhaustivo) dará mejor resultado en una primera instancia que una aproximación por aprendizaje.
      \item[b)] Aproximación por diseño
      \item[c)] ¿?
    \end{enumerate}
  \end{solucion}

  \begin{cuestion}
  Construir un problema de \emph{aprendizaje desde datos} para un problema de selección de fruta en una explotación agraria (ver transparencias de clase). Identificar y describir cada uno de sus elementos formales. Justificar las decisiones.
  \end{cuestion}

  \begin{solucion}

  \end{solucion}

  \begin{cuestion}
  Suponga un modelo PLA y un dato $x(t)$ mal clasificado respecto de dicho modelo. Probar que la regla de adaptación de pesos del PLA es un movimiento en la dirección correcta para clasificar bien $x(t)$.

  \end{cuestion}

  \begin{solucion}
    Definimos primero el término $y(t)w^T(t)x(t)$ como, informalmente,  el resultado de clasificar $x(t)$ con $w(t)$. Claramente es $>0$ si está bien clasificado y $<0$ si no lo está, ya que si está bien clasificado tenemos que $y(t) = sign(w^T(t)x(t))$, por lo cual su producto es positivo, y en caso contrario $y(t) \neq sign(w^T(t)x(t))$, lo cual su producto es negativo.

    Usando esto, veamos que $y(t)w^T(t+1)x(t) > y(t)w^T(t)x(t)$. Es claro ya que      si expandimos el primer miembro con la regla de actualización del \emph{PLA} vemos que:

    \[
    y(t)w^T(t+1)x(t) = y(t){\big( w(t)+y(t)x(t) \big)}^T x(t) = y(t)w^T(t)x(t) + \underbrace{{\norm{x(t)}}^2}_{>0}
    \]

    Por lo cual la desigualdad de arriba queda clara. Con esto deducimos que la clasificación de $x(t+1)$ siempre será \emph{mejor} que la de $x(t)$, ya que el valor de su clasificación es mayor, lo cual hace que pueda superar a $0$, implicando que el dato estaría bien clasificado. Podría no superar a $0$, pero en tal caso estaría más cerca de estar bien clasificado.
  \end{solucion}

  \begin{cuestion}
  Considere el enunciado del ejercicio 2 de la sección FACTIBILIDAD DEL APRENDIZAJE de la relación de apoyo.
  \begin{enumerate}
  \item[a)] Si $p=0,9$, ¿cuál es la probabilidad de que $S$ produzca una hipótesis mejor que $C$?
  \item[b)] ¿Existe un valor de $p$ para el cual es más probable que $C$ produzca una hipótesis mejor que $S$?
  \end{enumerate}
  \end{cuestion}

  \begin{solucion}
    Las soluciones son:
    \begin{enumerate}
      \item[a)] Queremos determinar cuál es la probabilidad de que el error fuera de la muestra sea menor para $S$ que para $C$:
        \begin{equation}
          \begin{split}
            P\left[E_{out}(S(\mathcal{D})) < E_{out}(C(\mathcal{D}))  \right] &=           P\left[E_{out}(h_1) < E_{out}(h_2)  \right] \\
            &= P\left[P\left[f(x) \neq h_1\right] < P\left[f(x) \neq h_2\right]  \right] \\
            &= P\left[P\left[f(x) = -1\right] < P\left[f(x) =+1\right]  \right] \\
            &= P\left[1-p < p \right] = P\left[0.1 < 0.9 \right] = 1
          \end{split}
        \end{equation}
      \item[b)] En este caso  razonamos igual que en el anterior, pero dándole la vuelta a la desigualdad y sin determinar $p$:
      \begin{equation}
        \begin{split}
          P\left[E_{out}(S(\mathcal{D})) > E_{out}(C(\mathcal{D}))  \right] &=           P\left[E_{out}(h_1) > E_{out}(h_2)  \right] \\
          &= P\left[P\left[f(x) \neq h_1\right] > P\left[f(x) \neq h_2\right]  \right] \\
          &= P\left[P\left[f(x) = -1\right] > P\left[f(x) =+1\right]  \right] \\
          &= P\left[1-p > p \right] = P\left[ 0.5 > p \right] = 1
        \end{split}
      \end{equation}
      Es claro que dicha probabilidad es $1$ si y solamente si $p < 0.5$
    \end{enumerate}
  \end{solucion}

  \begin{cuestion}
  La desigualdad de Hoeffding modificada nos da una forma de caracterizar el error de generalización con una cota probabilística
  \[
  \mathds{P}[|E_{out}(g) - E_{in}(g)| > \epsilon] \leq 2Me^{-2N \epsilon^2}
  \]
  para cualquier $\epsilon > 0$. Si fijamos $\epsilon=0,05$ y queremos que la cota probabilística $2Me^{-2N^2 \epsilon}$ sea como máximo $0,03$, ¿cuál será el valor más pequeño de $N$ que verifique estas condiciones si $M=1$? Repetir para $M=10$ y para $M=100$.

  \end{cuestion}

  \begin{solucion}
    Sólo tenemos que despejar $N$ de la ecuación para conseguir los valores deseados. Como queremos que la cota sea $0.03$, igualamos y despejamos:

      \[
      2Me^{-2N\epsilon^2} = 0.03
      \]

      \[
      e^{-2N\epsilon^2} = \frac{0.015}{M}
      \]

      \[
      -2N\epsilon^2 = log\left(\frac{0.015}{M}\right)
      \]

      \[
      N = \ceil{-log\left(\frac{0.015}{M}\right) \frac{1}{2\epsilon^2}}
      \]

      Ahora sólo tenemos que introducir $\epsilon = 0.05$ y los distintos valores de $M$.

      $$M =  1 \rightarrow N =  840$$
      $$M =  100 \rightarrow N =  1761$$
      $$M =  1000 \rightarrow N =  2222$$

  \end{solucion}

  \begin{cuestion}
  Consideremos el modelo de aprendizaje \emph{M-intervalos} donde $h: \mathbb{R} \rightarrow \{-1, +1\}$ y $h(x)=+1$ si el punto está dentro de cualquiera de $m$ intervalos arbitrariamente elegidos y $h(x)=-1$ en otro caso. ¿Cuál es el más pequeño punto de ruptura para este conjunto de hipótesis?

  \end{cuestion}

  \begin{solucion}

  \end{solucion}

  \begin{cuestion}
  Suponga un conjunto de $k^*$ puntos $x_1,x_2,...,x_{k^*}$ sobre los cuales la clase $H$ implementa $<2^{k^*}$ dicotomías. ¿Cuáles de las siguientes afirmaciones son correctas?
  \begin{enumerate}
  \item[a)] $k^*$ es un punto de ruptura
  \item[b)] $k^*$ no es un punto de ruptura
  \item[c)] todos los puntos de ruptura son estrictamente mayores que $k^*$
  \item[d)] todos los puntos de ruptura son menores o iguales a $k^*$
  \item[e)] no conocemos nada acerca del punto de ruptura
  \end{enumerate}

  \end{cuestion}

  \begin{solucion}
    La respuesta correcta es la e). Para llegar a una conclusión tendría que ser \emph{para todo conjunto de} $k^*$ \emph{puntos}
  \end{solucion}

  \begin{cuestion}
  Para todo conjunto de $k^*$ puntos, $H$ implementa $<2^{k^*}$ dicotomías. ¿Cuáles de las siguientes afirmaciones son correctas?
  \begin{enumerate}
  \item[a)] $k^*$ es un punto de ruptura
  \item[b)] $k^*$ no es un punto de ruptura
  \item[c)] todos los $k \geq k^*$ son puntos de ruptura
  \item[d)] todos los $k < k^*$ son puntos de ruptura
  \item[e)] no conocemos nada acerca del punto de ruptura
  \end{enumerate}
  \end{cuestion}

  \begin{solucion}
    Las respuestas correctas con a) y c). Como son todos los conjuntos, $2^{k^*}$ acota las dicotomías que implementa $H$. Además, cualquier valor mayor a dicho $k^*$ también lo acotará.
  \end{solucion}

  \begin{cuestion}
  Si queremos mostrar que $k^*$ es un punto de ruptura, ¿cuáles de las siguientes afirmaciones nos servirían para ello?:
  \begin{enumerate}
  \item[a)] Mostrar que existe un conjunto de $k^*$ puntos $x_1,\ldots,x_{k^*}$ que $H$ puede separar (\emph{shatter}).
  \item[b)] Mostrar que $H$ puede separar cualquier conjunto de $k^*$ puntos.
  \item[c)] Mostrar un conjunto de $k^*$ puntos $x_1,\ldots,x_{k^*}$ que $H$ no puede separar.
  \item[d)] Mostrar que $H$ no puede separar ningún conjunto de $k^*$ puntos.
  \item[e)] Mostrar que $m_H(k)=2^{k^*}$
  \end{enumerate}
  \end{cuestion}

  \begin{solucion}
    La solución es d). Si separara alguno, estaríamos en el caso de la cuestión 8. Si los separara todos, en el caso de la 9.
  \end{solucion}

  \begin{cuestion}
  Para un conjunto $H$ con $d_{VC}=10$, ¿qué tamaño muestral se necesita (según la cota de generalización) para tener un $95\%$ de confianza de que el error de generalización sea como mucho $0,05$?

  \end{cuestion}

  \begin{solucion}
    Tal y como se explica en las transparencias, se resolverá la cota del tamaño de la muestra iterativamente sobre la fórmula apropiada.

    \lstinputlisting{complejidad.py}

    El tamaño resultado de ejecutar este script es de 452957.
  \end{solucion}

  \begin{cuestion}
  Consideremos un escenario de aprendizaje simple. Supongamos que la dimensión de entrada es uno. Supongamos que la variable de entrada $x$ está uniformemente distribuida en el intervalo $[-1,1]$ y el conjunto de datos consiste en 2 puntos ${x_1,x_2}$ y que la función objetivo es $f(x)=x^2$. Por tanto el conjunto de datos completo es $\mathcal{D}={(x_1,x_1^2), (x_2,x_2^2)}$. El algoritmo de aprendizaje devuelve la línea que ajusta estos dos puntos como $g$ (i.e. $H$ consiste en funciones de la forma $h(x)=ax+b$).
  \begin{enumerate}
  \item[a)] Dar una expresión analítica para la función promedio $\overline{g}(x)$.
  \item[b)] Calcular analíticamente los valores de $E_{out}$, \emph{bias} y \emph{var}.
  \end{enumerate}

  \end{cuestion}

  \begin{solucion}

  \end{solucion}

  %%%% BONUS %%%%

  \begin{bonus}
    Considere el enunciado del ejercicio 2 de la sección ERROR Y RUIDO de la relación de apoyo.
    \begin{itemize}
      \item[a)] Si su algoritmo busca la hipótesis $h$ que minimiza la suma de los valores absolutos de los errores de la muestra,

      \[
      E_{in}(h) = sum_{n=1}^N{|h-y_n|}
      \]
      entonces mostrar que la estimación será la mediana de la muestra, $h_{med}$ (cualquier valor que deje la mitad de la muestra a su derecha y la mitad a su izquierda)

      \item[b)] Suponga que $y_n$ es modificado como $y_n + \epsilon$, donde $\epsilon \rightarrow \infty$. Obviamente el valor de $y_n$ se convierte en un punto muy alejado de su valor original. ¿Cómo afecta esto a los estimadores dados por $h_mean$ y $h_med$?
    \end{itemize}
  \end{bonus}

  \begin{solucion}

  \end{solucion}

  \begin{bonus}
    Considere el ejercicio 12.
    \begin{itemize}
      \item[a)] Describir un experimento que podamos ejecutar para determinar (numéricamente) $g(x)$, $E_{out}$, \emph{bias} y \emph{var}.

      \item[b)] Ejecutar el experimento y dar los resultados. Comparar $E_{out}$ con \emph{bias+var}. Dibujar en unos mismos ejes $g(x)$, $E_{out}$ y $f(x)$.
    \end{itemize}
  \end{bonus}

  \begin{solucion}

  \end{solucion}

\end{document}
